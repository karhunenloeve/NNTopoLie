\begin{figure}[t!]
\begin{tikzpicture}[
	bullet/.style={circle,
	fill,
	inner sep=2pt,
	label={#1}},
	>=latex,
	scale=0.5
]
	\draw[color=gray] (0,0) circle (85pt);
	\node[above, color=gray] at (2.5,2) {$\mathcal{S}^1$};
	\node[above] at (-5,-3) {(a)};
	\draw [black,line width=0.5pt,/.list={1/6,1/2,5/6}]
	(210:3) node[color = brick, bullet={below left:1}] (a) {}
	-- (90:3) node[color = brick, bullet={above:2}] (b) {}
	-- (-30:3) node[color = brick, bullet={below right:3}] (c) {}
	-- cycle;
	\node[above] at (-5,2) {$K_0 = \{\{1\}, \{2\}, \{3\}\}$};
	\node[above, fill=white] at (-4,1) {$K_1 = \{\{1,2\}, \{1,3\}, \{2,3\}\}$};
	\node[above, fill=white] at (-5,0) {Coefficients in $\mathbb{Z}/2\mathbb{Z}$};
\end{tikzpicture}
\begin{tikzpicture}[
	bullet/.style={circle,
	fill,
	inner sep=2pt,
	label={#1}},
	>=latex,
	scale=0.5
]
	\node[above, color=gray] at (2.5,2) {$\mathcal{S}^1$};
	\draw[color=gray] (0,0) circle (85pt);
	\node[above] at (-5,-3) {(b)};
	\draw [black,line width=0.5pt,->-/.list={1/6,1/2,5/6}]
	(210:3) node[color = brick, bullet={below left:1}] (a) {}
	-- (90:3) node[color = brick, bullet={above:2}] (b) {}
	-- (-30:3) node[color = brick, bullet={below right:3}] (c) {}
	-- cycle;
	\draw[line width=0.5pt,-{Latex[bend]}] (240:0.5) arc(240:-60:0.5);
	\node[above] at (-5,2) {$K_0 = \{\{1\}, \{2\}, \{3\}\}$};
	\node[above, fill=white] at (-3.9,1) {$K_1 = \{\{1,2\}, \{2,3\}\}, \{1,3\}\}$};
	\node[above] at (-5.7,0) {Coefficients in $\mathbb{R}$};
\end{tikzpicture}
\caption{Two triangulations of $\mathcal{S}^1$. We compute $H_0(\mathcal{S}^1) = \ker(\partial_0) / \Ima(\partial_1)$ for (a) and (b), see Sect. \ref{simcomplexhomtheory}. For (a): $\ker(\partial_0) = \text{span}(\{1\},\{2\},\{3\})$ and $\Ima(\partial_1)= \text{span}(\{\{2\}-\{1\}\},\{\{3\}-\{2\}\})$ as linearly independent elements. Observe that $\{1\}-\{3\}=\{2\}-\{1\}-(\{3\}-\{2\})$ holds. Thus $\ker(\partial_0) \simeq (\mathbb{Z}/2\mathbb{Z})^3$, $\Ima(\partial_1)=(\mathbb{Z}/2\mathbb{Z})^2$ and $H_0(\mathcal{S}^1;\mathbb{Z}/2\mathbb{Z}) = \mathbb{Z}/2\mathbb{Z}$. For (b): $\ker(\partial_0) = \text{span}(\{1\},\{2\},\{3\})$, but due to real coefficients the linear dependency is this time given by $\{2-1\} = \{3-2\}$. Thus $\ker(\partial_0) \simeq \mathbb{R}^3$ and $\Ima(\partial_1)=\mathbb{R}^2$, which yields the homology group $H_0(\mathcal{S}^1;\mathbb{R}) = \mathbb{R}$.}
\label{complexes}
\end{figure}